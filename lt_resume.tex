\documentclass[a4paper,11pt]{article}

\usepackage{fontspec}
%\setmainfont[Ligatures=TeX]{Times}

\usepackage{latexsym}
\usepackage[empty]{fullpage}
\usepackage{titlesec}
\usepackage{marvosym}
\usepackage[usenames,dvipsnames]{color}
\usepackage{verbatim}
\usepackage{enumitem}
\usepackage[hidelinks]{hyperref}
\usepackage{fancyhdr}
\usepackage{psvectorian}

\usepackage{geometry}
 \geometry{
 a4paper,
 total={170mm,257mm},
 left=10mm,
 right=10mm,
 top=10mm,
 bottom=10mm,
}

\pagestyle{fancy}
\fancyhf{} % clear all header and footer fields
\fancyfoot{}
\renewcommand{\headrulewidth}{0pt}
\renewcommand{\footrulewidth}{0pt}

\urlstyle{same}

\raggedbottom
\raggedright
\setlength{\tabcolsep}{0in}

% Sections formatting
\titleformat{\section}{
  \vspace{-4pt}\scshape\raggedright\large
}{}{0em}{}[\color{black}\titlerule \vspace{-5pt}]

%-------------------------
% Custom commands
\newcommand{\resumeItem}[2]{
  \item\small{
    \textbf{#1}{ #2 \vspace{-2pt}}
  }
}

\newcommand{\resumeSubheading}[4]{
  \vspace{-1pt}\item
    \begin{tabular*}{0.97\textwidth}{l@{\extracolsep{\fill}}r}
      \textbf{#1} & #2 \\
      \textit{\small#3} & \textit{\small #4} \\
    \end{tabular*}\vspace{-5pt}
}

\newcommand{\resumeSubItem}[2]{\resumeItem{#1}{#2}\vspace{-4pt}}
\newcommand{\resumeSubHeadingListStart}{\begin{itemize}[leftmargin=*] \renewcommand\labelitemi{$\circ$}}
\newcommand{\resumeSubHeadingListEnd}{\end{itemize}}

%-------------------------------------------
%%%%%%  CV STARTS HERE  %%%%%%%%%%%%%%%%%%%%%%%%%%%%

\begin{document}

%----------HEADING-----------------
\begin{tabular*}{\textwidth}[t]{l@{\extracolsep{\fill}} c r}
\textbf{\href{http://valaitis.net/}{\Large dr. Vytautas Valaitis}} & \textit{\textbf{C.V.}} & \href{mailto:vytautas@valaitis.net}{vytautas@valaitis.net}\\
  & & +370 673 37482 \\
\end{tabular*}

\begin{pspicture}(0,0)(0,0)%
  \rput[bl](0,0){\psvectorian[width=1cm,flip]{77}}
\end{pspicture}

%-----------EDUCATION-----------------
\begin{minipage}[t]{.56\textwidth}
\vspace{5pt}
\section{Išsilavinimas}
  \resumeSubHeadingListStart
    \resumeSubItem{Informatikos mokslų daktaras,}{Vilniaus universitetas, 2016.}
    \resumeSubItem{Informatikos magistras,}{Vilniaus universitetas, 2011.}
    \resumeSubItem{Programų sistemų bakalauras,}{Vilniaus universitetas, 2009.}
  \resumeSubHeadingListEnd
\end{minipage}%
\hfill
\begin{minipage}[t]{.42\textwidth}
\vspace{5pt}
\section{Kalbos}
  \resumeSubHeadingListStart
    \resumeSubItem{Lietuvių,}{gimtoji.}
    \resumeSubItem{Anglų,}{geras supratimas, rašymas ir kalbėjimas.}
  \resumeSubHeadingListEnd
\end{minipage}
%\vspace{-10pt}
%-----------EXPERIENCE-----------------
\section{Darbo patirtis}
  \resumeSubHeadingListStart
    \resumeSubItem{Vilniaus universitetas, programų sistemų katedra,}{asistentas, nuo 2017.}
    \resumeSubItem{Lietuvos Kompiuterininkų Sąjunga, Vilnius,}{projekto specialistas, nuo 2019.}
    \resumeSubItem{Monet Lt, UAB, Vilnius,}{tyrėjas vyr. programuotojas, nuo 2018.}
    \resumeSubItem{VšĮ Kosmoso mokslo ir technologijų institutas, Vilnius,}{jaunesnysis mokslo darbuotojas, 2018 - 2019.}
    \resumeSubItem{Vilniaus universitetas, programų sistemų katedra,}{lektorius, 2011 - 2017.}
    \resumeSubItem{Magma Solutions UAB, Vilnius,}{programuotojas inžinierius, 2017.}
    \resumeSubItem{Vilniaus universitetas, informatikos fakultetas,}{jaunasis mokslo darbuotojas, 2013 - 2015.}
    \resumeSubItem{VšĮ Inovatyvūs inžineriniai projektai, Vilnius,}{vyr. programuotojas, 2014 - 2015.}
    \resumeSubItem{UAB Teltonika, Vilnius,}{įterptinių sistemų programuotojas, 2009 - 2011.}
    \resumeSubItem{AB Lietuvos Energija, Vilnius,}{programuotojas, 2007 - 2008.}
    \resumeSubItem{UAB Prototechnikos Įranga, Vilnius,}{programuotojas, 2006 - 2007.}
  \resumeSubHeadingListEnd
%-----------PROJECTS-----------------
\vspace{-15pt}
\section{Projektai}
  \resumeSubHeadingListStart
    \resumeSubItem{Artificial Intelligence Skills for ICT Professionals,}{2019-1-BE01-KA202-050425, 2019 - 2022.}
    \resumeSubItem{Profesijos ir suaugusiųjų mokytojų kvalifikacijos tobulinimo sistemos plėtra,}{09.4.2-ESFA-V-715-01-0001, 2019 - 2020.}
    \resumeSubItem{Inovatyvių skaitmeninės buhalterijos technologijų kūrimas,}{J05-LVPA-K-04, 2018 - 2020.}
    \resumeSubItem{Didelės apimties buhalterinių duomenų kaupimo ir analizės technologijų kūrimo galimybių studija,}{01.2.1-MITA-T-851-01-0022, 2018 - 2019.}
    \resumeSubItem{Hybrid UAV Kolibris,}{KAM-01-08, 2016.}
    \resumeSubItem{High-dimensionality and small data size problems in classification of biomedical and financial data,}{MIP-13011, 2013 - 2015.}
    \resumeSubItem{Lituanica SAT-1,}{2012 - 2014.}
  \resumeSubHeadingListEnd
%-----------PUBLICATIONS-----------------
\vspace{-15pt}
\section{Publikacijos}
    \resumeSubHeadingListStart
    \resumeSubItem{Valaitis, V. et al Minimizing Hexapod Robot Foot Deviations Using Multilayer Perceptron.\\}{International Journal of Advanced Robotic Systems. Vol. 12, 2015.}
    \resumeSubItem{Valaitis, V. et al Piezoelectric Force Sensors for Hexapod Transportation Platform.\\}{Transport: Special Issue on Smart and Sustainable Transport. Vol. 30, No. 3, 2015.}
    \resumeSubItem{Valaitis, V. et al A Price We Pay for Inexact Dimensionality Reduction.\\}{Bioinformatics and Biomedical Engineering. p. 289-300, 2015.}
    \resumeSubItem{Valaitis, V. Learning Inverse Kinematics Problem in Changing Task Environment.\\}{The 12th Scandinavian AI conference. Vol 257, p. 299-302, 2013.}
  \resumeSubHeadingListEnd
%-----------CONFERENCES-----------------
\vspace{-15pt}
\section{Konferencijos}
  \resumeSubHeadingListStart
    \resumeSubItem{Numerical Computations: Theory and Algorithms, Crotone, Italy, 2019.\\}{Learning Aerial Image Similarity using Triplet Networks.}
    \resumeSubItem{The 3rd IEEE Workshop  on Bio-Inspired Signal and Image Processing, Vilnius, Lithuania, 2014.\\}{Multi-agent Neural Network Approach on Inverse Kinematics Problem in Changing Task Environment.}
    \resumeSubItem{Numerical Computations: Theory and Algorithms, Falerna, Italy, 2013.\\}{Learning Motion Patterns of Robotic Arm.}
    \resumeSubItem{The 12th Scandinavian AI conference, Aalborg, Denmark, 2013.\\}{Learning Inverse Kinematics Problem in Changing Task Environment.}
  \resumeSubHeadingListEnd
\vfill 
\begin{pspicture}(0,0)
  %\psframe[fillstyle=solid,fillcolor=blue!20,linewidth=3pt](0,0)(\textwidth,1)
  \rput[b](.5\textwidth,0){\psvectorian[width=2cm]{70}}
\end{pspicture}
\end{document}
